\documentclass[11pt, a4paper]{article}

\usepackage{fontspec}
\usepackage{geometry}
\usepackage{lastpage}
\usepackage{fancyhdr}
\usepackage[hidelinks]{hyperref}
\usepackage[normalem]{ulem}
\usepackage{soul}
\usepackage{multicol}
\usepackage[dvipsnames]{xcolor}

\geometry{
  top=1cm,
  bottom=2cm,
  left=1cm,
  right=1cm,
  marginparsep=4pt,
  marginparwidth=1cm
}

\definecolor{lgrey}{rgb}{0.8, 0.8, 0.8}

\renewcommand{\headrulewidth}{0pt}
\pagestyle{fancyplain}
\fancyhf{}
\lfoot{\color{lgrey}2021-04-09}
% \rfoot{\color{lgrey}page \thepage\ of \pageref{LastPage}}

\setlength{\parindent}{0pt}
\setlength{\parskip}{0pt}

\usepackage{xunicode}
\defaultfontfeatures{Mapping=tex-text}

\setromanfont{YaleNew}

\begin{document}

\begin{center}
\textbf{Introduction to Data Science (LING/MATH/DSST 289) --- Fall 2021}
\end{center}

\noindent
\begin{tabular}{ l l }
\textbf{Instructor:} &  \textbf{Taylor Arnold} \\
E-mail: & \texttt{tarnold2@richmond.edu} \\
Website: & \texttt{https://statsmaths.github.io/dsst289-f21}
\end{tabular}

\vspace{0.5cm}

\textbf{Description:} \vspace{6pt}

Data science is an interdisciplinary field concerned with creating knowledge
from data and communicating those results. Data
science needs to be learned \textit{by doing} data science. We
will focus in this course on collecting data and creating data-driven stories
within a variety of different domains. We will use a mix of
manual techniques---including creating hand-drawn graphics---and digital methods
using freely-available programming languages. At the end of the
semester, students will be able to collect and use data in order to address
important social, cultural, and scientific questions.

\bigskip

\textbf{Books and Materials:} \vspace{6pt}

The following texts and materials, all available in the UR bookstore, are
required for the course:

\begin{itemize}
\setlength\itemsep{-0.25em}
\item \textit{Storytelling with Data}, by Cole Nussbaumer Knaflic (2015),
ISBN-13: 978-1119002253.
\item \textit{Dear Data}, by Giorgia Lupi and Stefanie Posavec (2016),
ISBN-13: 978-1616895327.
\item One 8.5''x11'' (or slightly larger) spiral-bound, unlined sketchbook,
with 80+ pages.
\item A small set of colored pencils and ruler; these may shared with other
students.
\end{itemize}

We will also use several different computer programs for creating data and
visualizations. All of these are free to use and can be accessed through any
modern web browser.

\bigskip

\textbf{Course Format:} \vspace{6pt}

Most class meetings will have an assigned homework task, such as a
reading or data collection activity, that should be completed before class.
These will not be formally graded, but their
completion is part of the participation expectations (see below). During class,
most of our time will be devoted to actively working on tasks individually or in
small groups.\\

Students will also complete a number of longer projects focused on different
forms of data storytelling. The projects will give a chance to deepen and
extend the techniques presented during the in-class activities. Time
permitting, students will have a chance to present their projects during class.
Projects will be given a grade of either Satisfactory or Unsatisfactory.
There will be a possibility to resubmit one unsatisfactory project during the
semester.

\bigskip

\textbf{Course Engagement:} \vspace{6pt}

All students enrolled in the course are expected to be fully engaged in class
meetings. Engagement includes arriving on time, bringing required materials,
completing any assigned homework, and being attentive and engaged with all
class activities. While attendence at all class meetings is preferable, in the
understanding that unavoidable conflicts and circumstances (job interviews,
illness, etc.) arise from time to time, students will not be penalized for
their first two absences.

\bigskip

\textbf{Learning Objectives, Self Evaluation, and Final Grades:} \vspace{6pt}

Students will be asked to reflect on their own learning objectives and
expectations for the course during the first few weeks of the semester. During
the last week of the course, a self evaluation describing the extent to which
these objectives have been achieved through the semester will be submitted.\\

All students who have earned satisfactory grades on of the projects, attended
and fully engaged with all but at most two classes, and have completed the
learning objects and self evaluation will be awarded a minimum grade of a B.
Grades of B, B+, A-, A, and A+ will be given according to each student's
self-assessment in light of their learning objects, projects and course
participation. Students failing to achieve these minimum course expectations
may result in a lower grade, up to an including a failing grade.

\bigskip

\textbf{Other Resources:} \vspace{6pt}

Other resources, including information about office hours and a full list of
campus offices offering support to UR students, as well as more detailed
instructions for each of the course components are posted and kept updated on
the class website.

\end{document}
