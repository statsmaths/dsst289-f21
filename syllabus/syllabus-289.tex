\documentclass[11pt, a4paper]{article}

\usepackage{fontspec}
\usepackage{geometry}
\usepackage{lastpage}
\usepackage{fancyhdr}
\usepackage[hidelinks]{hyperref}
\usepackage[normalem]{ulem}
\usepackage{soul}
\usepackage{multicol}
\usepackage[dvipsnames]{xcolor}

\geometry{
  top=2cm,
  bottom=2cm,
  left=2cm,
  right=2cm,
  marginparsep=4pt,
  marginparwidth=1cm
}

\definecolor{lgrey}{rgb}{0.8, 0.8, 0.8}

\renewcommand{\headrulewidth}{0pt}
\pagestyle{fancyplain}
\fancyhf{}
\lfoot{\color{lgrey}2021-07-11}
% \rfoot{\color{lgrey}page \thepage\ of \pageref{LastPage}}

\setlength{\parindent}{0pt}
\setlength{\parskip}{0pt}

\usepackage{xunicode}
\defaultfontfeatures{Mapping=tex-text}

\setromanfont{YaleNew}

\begin{document}

\begin{center}
\textbf{Introduction to Data Science (LING/MATH/DSST 289) --- Fall 2021}
\end{center}

\noindent
\begin{tabular}{ l l }
\textbf{Instructor:} &  \textbf{Taylor Arnold} \\
E-mail: & \texttt{tarnold2@richmond.edu} \\
Website: & \texttt{https://statsmaths.github.io/dsst289-f21}
\end{tabular}

\vspace{0.5cm}

\textbf{Description:} \vspace{6pt}

Data science is an interdisciplinary field concerned with creating knowledge
from data and communicating those results. Data science needs to be learned
\textit{by doing} data science. We will focus in this course on collecting data
and creating data-driven stories within a variety of different domains through
the use of statistical software. At the end of the semester, students will be
able to collect and use data in order to address important social, cultural,
and scientific questions.

\bigskip

\textbf{Exams:} \vspace{6pt}

There will be three take-home exams given throughout the semester. Any passive
resources (i.e., notes, books, and software documentation) may be consulted
during the exame, but students must work on the material on their own. Exams
are handed out on Thursdays (see the course website for the exact dates) and
are due the following Tuesday before class. Exams will be graded out of 95
points.\\

In the event that a student's performance on a take-home exam is significantly
divergent from their performance in class, a follow-up oral exam may be
requested. The exam grade may be changed if skills displayed in the oral exam
continue to differ from the written exam.

\bigskip

\textbf{Course Engagement:} \vspace{6pt}

Students enrolled in the course are expected to be fully engaged in
class meetings. Engagement includes arriving on time, bringing required
materials and being attentive and engaged with all class activities. A short
reflection is due at the end of the semester in which students explain their
engagement with the course. A grade for this reflection will be given in light
of efforts given throughout the semester.

\bigskip

\textbf{Final Grades:} \vspace{6pt}

A final numeric grade is determined by taking the average of the three exams,
and the course engagement grade (25\% each). Letter grades are assigned as
follows: A (93--100), A- (90--92), B+ (87--89), B (83--86), B- (80--82),
C+ (77--79), C (73--76), C- (70--72), and F (0--69). Grades of A+ and D are not
normally given.\\

In addition to impacting the course engagement score, excessive absences may
result in a further reduction in the overall course grade. Missing $3$ classes
will result in a one mark reduction (i.e., a B becomes a B-), missing $4$
classes results in a full letter reduction, and missing $5$ or more courses
results in a automatic failing grade.

\bigskip

\textbf{Homework, Final Exam, and Course Texts:} \vspace{6pt}

The class has no graded homework, required course texts, or final examination.
Notes, readings, and additional practice questions will be posted on the class
website throughout the semester.

\bigskip

\textbf{Other Resources:} \vspace{6pt}

Other resources, including information about office hours and a full list of
campus offices offering support to UR students, as well as more detailed
instructions for each of the course components are posted and kept updated on
the class website.

\end{document}
